\chapter{Implementierung}

Kann mit 5. Zusammenfallen. Manchmal eignet es sich, 2 Abstraktionsschritte zu machen (Realisierung und Implementierung getrennt).  
  
Generell für 5. Und 6.: Wenig Quellcode (wenn überhaupt)! Maximal 2/3 Seite und immer begleitet von Erklärungen, was zu sehen ist. Kommentare im Quellcode sind nicht ausreichend. Dies gilt für UML Diagrammen analog.  


    \section{Setup}
        \subsection{Entwicklungsumgebung}


        \subsection{Microsoft Visual Studio Code}


        \subsection{pipenv \- Python Package Manager}
            Die Applikation nutzt den Package Manager pipenv. Dieser bietet die Möglichkeit, ein projektspezifisches Dokument über alle Abhängigkeiten hinweg zu erstellen und im Projekt selbst zu speichern. So können andere Entwickler Abhängigkeiten leicht installieren und müssen dies nicht auf Systemebene tun, wo es ggf. zu Konflikten mit anderen Projekten kommen könnte.  
            Um alle Abhängigkeiten einzusehen, pipenv \cite{pipenv} installieren und das coaching\_bot\/Pipfile entsprechend der Dokumentation nutzen, um alle automatisch zu installieren.


        \subsection{Konstanten und Schlüssel}
            Der Coaching Bot hat einige Abhängigkeiten zu Umsystemen, die Zugangsdaten voraussetzen. Diese sind im Repository \cite{repo} aus Sicherheitsgründen abstrahiert und können durch kleine Anpassungen adaptiert werden. Entsprechende Vorlagen sind unter \emph{\_constants} zu finden.


    \section{Coaching Bot Herzstück \- main.py \- State Machine}
        Die main.py ist das Herzstück des Coaching Bots.

        Sie importiert alle Handler\-Funktionen, authentifiziert sich durch den entsprechenden API\-Schlüssel und beinhaltet den Dispatcher, an dem wiederum Conversation\- sowie CommandHandler hängen. Darüber hinaus startet sie den Bot und aktualisiert die Handler in regelmäßigen Abständen via dem Updater.

        Im Folgenden werden die CommandHandler für den Coaching Bot kurz aufgeführt:
        
        \subsection{Dispatcher}
            Dispatcher liefern Nachrichten an den User aus. Pro Bot gibt es meist einen Dispatcher. Der Coaching Bot hat aber mehrere für mehrere Konversationsstränge. Command\- und ConversationHandler werden an diese übergeben.


        \subsection{ConversationHandlers}
            ConversationHandler kontrollieren den Konversationsfluss zwischen dem User und dem Bot. Pro Bot kann es mehrere ConversationHandler geben. Der CoachingBot hat aber nur einen \- den conv\_handler. Der ConversationHandler koordiniert alle CommandHandler.


        \subsection{CommandHandler}
            CommandHandler nehmen Nutzereingaben via eines CallbackContexts entgegen, prüfen diese auf vordefinierte Kriterien und führen prädefinierte Funktionen \- sog. Handler Functions aus.


        \subsection{start und cancel \- Konversation starten und stoppen}
            Der in dieser Applikation umfangreichste ConversationHandler umfasst zwei Commands: /start und /cancel. Solange der Bot ausgeführt wird, lässt sich eine Konversation mit ihm über den Befehlt /start starten und via /cancel beenden. Zur Funktionsweise von /start und /cancel, siehe start.py und cancel.py unter Handler Funktionen. 


        \subsection{delete \- Nutzerdaten löschen}
            Über den Befehl /delete wird der CommandHandler delete ausgeführt. Zur Funktionsweise von /delete, siehe delete.py unter Handler Funktionen. 


        \subsection{help \- Hilfe ausgeben}
            Über den Befehl /help wird der CommandHandler help ausgeführt. Zur Funktionsweise von /help, siehe help.py unter Handler Funktionen. 


        \subsection{summary \- Zusammenfassung ausgeben}
            Über den Befehl /summary wird der CommandHandler summary ausgeführt. Zur Funktionsweise von /summary, siehe summary.py unter Handler Funktionen. 


        \subsection{status \- Status Quo ausgeben}
            Über den Befehl /status wird der CommandHandler status ausgeführt. Zur Funktionsweise von /status, siehe status.py unter Handler Funktionen. 


    \section{Handler Funktionen}
        Handler\-Funktionen (siehe coaching\_bot/handler\_functions in \cite{repo}) sind Funktionen, die auf Eingaben reagieren, die via CallbackContext vom User an den Bot gesendet werden und bestimmten Kriterien entsprechen. Diese Kriterien werden direkt in der main.py in einem der CommandHandler definiert. Im Folgenden gehen wir detailliert auf die einzelnen Handler\-Funktionen ein, beschreiben deren Umfang und Aufbau und erklären ihre Funktionsweise. 

        \subsection{start.py}
            Die Methode start in der start.py fungiert als Eingangstor für jeden User. Wann immer der Befehl /start an den Bot schickt wird, löst der CommandHandler die Methode start aus.

            Zunächst wird geprüft, ob es eine Datenbank gibt. Ist dies der Fall, wird geprüft, ob der Nutzer, der die Methode ausgelöst hat, bereits in der Datenbank existiert. 
            
            Ist dies der Fall, gibt die Methode eine Willkommen\-zurück\-Nachricht aus und differenziert zwischen unterschiedlichen Reaktionen auf verschiedene Zustände:
            1. Befindet der Nutzer sich im Zustand \emph{SUMMARY}, hat also bereits alle Fragen beantwortet, aber noch keinen Termin vereinbart, so werden in diesem Zustand sinnvolle Optionen empfohlen. Der Nutzer kann sich den Status seiner Bewerbung ausgeben lassen, die Zusammenfassung erneut beantragen oder alle seine Daten löschen. 
            2.  Befindet der Nutzer sich im Zustand \emph{APPOINTMENT}, hat aber noch keinen Termin vereinbart, die Zusammenfassun aber bereits erhalten, erhält er zusätzlich zur Option, sich die Zusammenfassung erneut ausgeben zu lassen und so die Terminfindung zu starten, nur die Status\- und Lösch\-Optionen. Natürlich kann der Nutzer auch manuell alle Befehle jederzeit eingeben, aber die Tastatur ist so für Optionen vordefiniert, dass der Nutzer in eine bestimmte Richtung gelenkt wird. Nach der Ausgabe dieser Nachricht, beendet der ConversationHandler die Kommunikation. 
            3. Befindet der Nutzer sich im Zustand \emph{APPOINTMENT} und hat bereits einen Termin vereinbart, so werden Informationen zu dem Termin aus der Datenbank abgerufen und direkt ausgegeben. Auch in diesem Fall wird die Konversation nun beendet, da keine weiteren Interaktionen mit dem Nutzer vorgesehen sind.

            Egal welche Option die Methode wählt, der Nutzer wird immer in den Konversationsfluss zurückgeführt und zwar genau vor der Frage, die zuletzt nicht beantwortet wurde. Eine Frage zu überspringen gilt dabei auch als Beantwortung. Dazu wird die Datenbank abgefragt und der Wert aus \emph{state} für die entsprechende User\-ID an den ConversationHandler weitergegeben. Dieser präsentiert als Antwort darauf die nächste Frage im Konversationsfluss.

            Treffen all diese Konditionen nicht zu, wurde der Nutzer also nicht in der Datenbank gefunden, so startet der Bot ganz normal mit einer Begrüßung, nachdem initiale Daten von der Telegram\-Instanz des Nutzers abgefragt und in die Datenbank geschrieben wurden. 

            Ist die Nachricht an den Nutzer ausgeliefert, aktualisiert der Bot den Zustand für den Nutzer in der Datenbank, damit der Bot weiß, welche Fragen der Nutzer schon beantwortet hat und er den Nutzer bei einer Rückkehr wieder am richtigen Punkt in den Konversationsfluss einfügen kann.

            Bevor der Bot den Nutzer zur nächsten Stufe weiterleitet, speichert er noch einen Zeitstempel, damit man nachvollziehen kann, wann der Nutzer seinen Prozess begonnen hat.


        \subsection{bio.py}
            \paragraph{Methode bio}
                Die Methode bio in der bio.py speichert die Text\-Eingabe eines Nutzers als erste Nutzereingabe nach dem /start\-Befehl. Sie repräsentiert besonders gut den Aufbau der Handler\-Funktionen, weil sie über das Speichern und weiterleiten keine weiteren Features besitzt. Daher erläutern wir hier den Aufbau der Handler\-Funktionen hier beispielhaft für alle anderen Handler\-Funktionen:

                \begin{pyverbatim} 

                # Stores the information received and continues on to the next state
                def bio(update: Update, context: CallbackContext) -> int:
                    
                    user_id = update.message.from_user.id
                    bio_message = update.message.text
                    
                    logger.info(f'+++++ Bio of user {user_id}: {bio_message} +++++')

                    # write bio to DB
                    insert_update(user_id, 'bio', bio_message)

                    # reply keyboard for next state
                    update.message.reply_text(
                        'What a story! We will definately pick that up in our first session!\n\n' + \
                        'Ok - now let\'s get some basics down: \n' + \
                        states.MESSAGES[states.GENDER],
                        reply_markup=states.KEYBOARD_MARKUPS[states.GENDER],
                        )

                    # save state to DB
                    insert_update(user_id, 'state', states.GENDER)
                    return states.GENDER
                
                \end{pyverbatim} 

                Zunächst werden ein Update\- und ein CallbackContext\-Objekt an die Handler\-Methode übergeben. Zurückgegeben wird der Datentyp int, da die State\-Machine am Ende der Methode wissen muss, in welchen Zustand der Nutzer als nächstes geschickt werden soll. 
                Innerhalb der Methode werden user\_id und bio\_message aus dem Update\-Objekt gespeichert, da man diese beiden informationen gleich weiterverwenden möchte. Die bio\_message ist in diesem Fall die Text\-Eingabe, die an den Bot nach der letzten Stufe (start) übermittelt wurde. Die user\_id ist die Telegram\-ID des jeweiligen Nutzers. 
                Nach der Ausgabe eines einfachen Log\-Eintrags dazu, welcher Nutzer gerade welche Nachricht gesendet hat, wird der Datenbankeintrsg des Nutzers um die soeben empfangene Nachricht erweitert. (Funktionalität der inser\_update Methode folgt unter Abschnitt insert\_update.py)
                Nun kann die für den Nutzer sichtbare Reaktion auf die Nachricht erfolgen. Die update.message.reply\_text Methode erlaubt es uns, dem Nutzer einen beliebigen String sowie eine für diese Nachricht individuelles Antwort\-Tastatur auszugeben. Die zu übergebenden Parameter sind für die meisten reply\_text\-Instanzen in der states.py zentral gespeichert, um sich innerhalb der einzelnen Handler\-Funktion von möglichst viel Inhalt zu abstrahieren.
                Ist die Ausgabe an den Nutzer erfolgt, bleibt noch die Aktualisirung des Zustands des Nutzers in der Datenbank, gefolgt von der Übergabe des nächsten Zustands an den ConversationHandler.

                Der Aufbau aller weiteren Handler\-Funktionen ähnelt der Methode bio sehr stark. Auf Erweiterungen und Anpassungen wird in den entsprechenden Abschnitten eingegangen. Die Methode leitet in den Zustand GENDER.

            \paragraph{Methode skip\_bio}
                Die Methode skip\_bio in der bio.py wird durch den Befehl /skip ausgelöst. Dieser Befehl ist in jedem Zustand spezifisch für den CommandHandler einer Stufe definiert und hat in jedem Zustand einen anderen Effekt. In diesem Fall, wird die Methode skip\_bio aus der bio.py aufgerufen. Auch für die Methode skip\_bio gilt, dass sie den Aufbau der skip\-Methoden gut repräsentiert. Daher auch hier wieder eine detaillierte Erklärung:

                \begin{pyverbatim} 
                # Skips this information and continues on to the next state
                def skip_bio(update: Update, context: CallbackContext) -> int:
                    
                    user_id = update.message.from_user.id

                    logger.info(f'00000 No bio submitted by {user_id} 00000')

                    # alternative message
                    update.message.reply_text(
                        'Alright. No problem. I know, it can be uneasy to share at first. If you would like, I can offer you a free "gettin to know each other" phone call once you have finished the sign up.',
                        reply_markup=ReplyKeyboardRemove(),
                        )

                    # reply keyboard for next state
                    update.message.reply_text(
                        states.MESSAGES[states.GENDER],
                        reply_markup=states.KEYBOARD_MARKUPS[states.GENDER],
                        )    

                    # save state to DB
                    insert_update(user_id, 'state', states.GENDER)
                    return states.GENDER
                
                \end{pyverbatim} 

                Der Aufbau ähnelt der bio\-Methode. Allerdings liegt hier ein reduzierter Umfang und natürlich eine andere Nachricht an den Nutzer vor. So gibt der Logger nur aus, dass keine Nachricht eingegangen ist. Ein Update der Datenbank fällt weg, da der Nutzer keine neuen Informationen angegeben hat. Hier werden zwei reply\_text\-Methoden verwendet. Die Erste dient dazu, eine auf diese skip\-Methode individuelle Nachricht zu übermitteln. Die Zweite ähnelt der Methode aus der bio\-Funktion. Sie übermittelt die Aufforderung zur Eingabe der Information für die nächste Stufe und zeigt die entsprechende Tastatur an. Der Rest der Methode gleicht ihrer Schwester.


        \subsection{gender.py}
            \paragraph{Methode gender}
                Die einzige Besonderheit der gender\-Methode aus gener.py liegt in der Differenzierung der Datenbankoperationen, die als Resultat der vordefinierten Antwort des Nutzers ausgelöst werden. Die Optionen "Gentleman", "Lady" und "Unicorn" resultieren in einem nüchternen Datenbankeintrag: "male", "female", "diverse". Die Methode leitet in den Zustand "BIRTHDAY".
            
            \paragraph{Methode skip\_gender}
                Keine Besonderheiten. 
        

        \subsection{birthdate.py}
            \paragraph{Methode birthdate}
                Der Bot arbeitet hier erstmals mit Input\-Validierung. Dazu wird die Nutzereingabe zunächst an die Methode validate\_birthdate übergeben und auf die Bewertung des Inputs gewartet. Entspricht der Input dem prädefinierten Format, fährt der Bot wie gewöhnlich fort und übergibt den nächsten Zustand zurück an den ConversationHandler. Ist dies jedoch nicht der Fall, so wird eine entsprechende Nachricht an den Nutzer ausgegeben. Da der ConversationHandler erst dann zur nächsten Stufe geht, wenn er von der Methode birthdate den entsprechenden Zustand zurückerhalten hat, entsteht hier ein loop, der entweder durch eine gültige Eingabe oder eine der Meta\-Funktionen gebrochen werden kann.
        
        
        \subsection{email.py}
            \paragraph{Methode email}
                Wie die Methode auch schon, nutzt birthdate Input-Validation - dieses mal, um zu prüfen, ob eine gültige E\-Mail\-Adresse eingegeben wurde. 

            \paragraph{Methode skip\_email}
                Die Methode email ist die einzige Methode, die nicht übersprungen werden kann. Ohne eine gültige E\-Mail\-Adresse des Nutzers können wichtige Folgefunktionen des Bots nicht genutzt werden und der Sinn und Zweck (Terminvereinbarung) ist nicht möglich. Daher ist die Methode skip\_email so gestaltet, dass sie keinen Zustand zurückgibt, sondern den Nutzer im aktuellen Zustand belässt, bis dieser entweder eine gültige Adresse eingegeben oder einen alternativen Befehl abgesetzt hat, der ebenfalls das Ende der Konversation zufolge hat. So steht es dem Nutzer frei, die Konversation jederzeit zu beenden. 
        
            
        \subsection{telephone.py}
            \paragraph{Methode telephone}
                Die Methode telephone funktioniert exakt gleich wie die Methode email.

            \paragraph{Methode skip\_telephone}
                Die Methode skip\_telephone bietet dem Nutzer an, den Kontakt mit dem Anbieter alternativ via dem auf der Internetseit verfügbaren Webformular zu suchen. Dies ist generell für alle Informationen möglich, die an den Bot übergeben werden (allerdings lag der Beweggrund für die Erstellung des Bots darin, eine Alternative zum klassischen Kommunikationsmedium Web\-Formular zu bieten).
        
        
        \subsection{location.py}
            \paragraph{Methode location}
                Der Nutzer hat hier die Möglichkeit, seinen Standort anzugeben. Dazu wird die bereits in Telegram vorhandene Funktion zur Standortfreigabe genutzt. 
                Am Ende der Methode, bevor der Bot zur nächsten Stufe \emph{PHOTO} weitergeht, sendet der Bot ein Bild von sich selbst, um den Nutzer dazu anzuregen, auch sein Bild von sich zu teilen. Dazu wird ein einfaches JPG verwendet, das im Repository des Bots gespeichert ist. Der Pfad kann aber auch leicht an jede andere Ressource angepasst werden.

            \paragraph{Methode skip\_location}
                Keine Besonderheiten.
        

        \subsection{photo.py}
            \paragraph{Methode photo}
                Entscheidet der Nutzer sich, ein Bild mit dem Bot zu teilen, so nimmt die Methode photo dieses entgegen und speichert es in einem Ordner, der je nach System gewählt werden kann. Hier wurde ein Ordner im gleichen Verzeichnis gewählt, in dem der Bot existiert. Um Bilder später wieder zuordnen zu können, wird der Dateiname jedes Bildes auf die ID des jeweiligen Nutzers gesetzt, bevor es gespeichert wird.

            \paragraph{Methode skip\_photo}
                Keine Besonderheiten.
        

        \subsection{summary.py}
            \paragraph{Methode summary}
                In der Methode summary kommt alles zusammen. Der Nutzer hat nun alle Angaben gemacht oder übersprungen. Die Methode beginnt damit, eine Reihe von Informationen von der Datenbank abzufragen und in variablen zu speichern. Es werden nur Informationen abgefragt, die auch in der auszugebenden Nachricht genutzt werden sollen. Direkt darauf wird der String für die Nachricht zusammengebaut und gespeichert. Es folgt eine einfache Danke-Nachricht an den Nutzer, bevor die eigentliche Logik der Methode beginnt.
                
                Nun gibt es mehrere Szenarien aus Nutzersicht: Der Bot prüft, ob der Nutzer bereits einen Termin vereinbart hat.
                
                \begin{enumerate}
                \item Option A:  Der Nutzer ist bis zum Zustand \emph{SUMMARY} gekommen, hat die Zusammenfassung ausgegeben bekommen, dann aber keinen Termin vereinbart und den Chat verlassen. Der Nutzer kehrt nun zum Chat zurück und gibt erneut /start ein, um seine Konversation wieder aufzunehmen. Der Bot findet den Nutzer in der Datenbank und leitet an die Stufe \emph{SUMMARY} weiter. 
                Diesen Fall behandelt der Bot wie folgt:
                    \begin{enumerate}
                        \item Ist dem nicht der Fall, möchte er dem Nutzer jetzt drei mögliche Terminvorschläge unterbreiten und startet dazu die Terminfindung (siehe Abschnitt \ref*{Kalender}). % Verlinken auf anderen Abschnitt
                        \item Sobald die Termine zurückkommen, präsentiert der Bot diese dem Nutzer in Form eines entsprechenden Tastatur-Layouts. Das Layout ist dabei dynamisch und generiert sich bei jeder Abfrage neu.
                    \end{enumerate}
                \item Option B: Der Nutzer ist bis zum Zustand \emph{SUMMARY} gekommen und hat bereits einen Termin vereinbart.
                Diesen Fall behandelt der Bot wie folgt:
                Der Bot fragt den Termin von der Datenbank ab und gibt ihn in einer Nachricht an den Nutzer zurück. Gleichzeitig schlägt er dem Nutzer weitere mögliche Befehle vor, die an dieser Stelle Sinn machen und beendet die Konversation.
                \end{enumerate}

            Schließlich wird die Methode confirmation\_mail aufgerufen, die die gleiche Zusammenfassung nochmals per E\-Mail an die Adresse des Nutzers sendet (siehe Abschnitt confirmation\_mail.py) und die Information darüber, dass an diesen Nutzer bereits eine E\-Mail gesendet wurde wird neben den üblichen Abschlussbefehlen in der Datenbank gespeichert.

                    
        \subsection{confirmation\_mail.py}
                \paragraph{Methode confirmation\_mail}
                    Um dem Nutzer die Zusammenfassung in Form einer E-Mail zukommen zu lassen, muss diese zunächst zusammengesetzt werden. Das einfach zu bewerkstelligung, bietet uns die Bibliothek mime. \cite{mime} Daneben wird die smtplib-Bibliothek genutzt, um eine sichere Verbindung zu einem Mail-Server aufzubauen. \cite{smtplib}

                    Erforderliche Zugangsdaten werden außerhalb der Methode confirmation\_mail aus den constants abgefragt und für die Verwendung innerhalb der Methode gespeichert. So werden diese nicht bei jedem Methodenaufruf erneut abgerufen.

                    Die Methode bekommt Empfänger\-Name sowie \-Adresse und die Zusammenfassung aus der Methode summary übergeben. 
                    Über die smtplib wird ein Server\-Objekt erstellt. Gegenüber diesem Server authentifiziert sich der Bot nun via Benutzername und Passwort.

                    War die Authentifizierung erfolgreich, wird die eigentliche Nachricht zusammengesetzt. Dazu benötigt werden vier Bauteile: Sender\-Adresse, Empfänger\-Adresse, der Betreff und die Nachricht selbst.

                    Die Nachricht wird zuerst via der Methode attache() zusammengesetzt, um dann aus dem vordefinierten String ein message\-Objekt zu bauen.

                    Schließlich kann die E\-Mail via der Methode sendmail() unter Verwendung des zuvor angesprochenen Mail\-Servers verschickt werden.
                    
                    Die Verbindung zum Server wird getrennt.

        
        \subsection{help.py}
            \paragraph{Methode help}
            Die Methode help setzt ein Dictionary aus einer Liste an Befehlen zusammen, das dann ausgegeben werden kann. Ein Dictionary bietet die Möglichkeit, die Hilfe jederzeit einfach anzupassen, um Elemente zu erweitern oder zu reduzieren, ohne die Logik, über die die Hilfe ausgegeben wird, zu beeinflussen. Dazu wird die collections\-Bibliothek eingebunden, die es erlaubt, ein geordnetes Dictionary zu erstellen. Nachdem der String für die Hilfe zusammengesetzt ist, wird dieser einfach via der Methode send\_message() ausgegeben.)


        \subsection{states.py}
            \paragraph{STATES}
                Die State Machine muss zu jeder Zeit wissen, welche Zustände es gibt und in welcher Reihenfolge diese existieren. Dazu nutzt der python\-conversation\-bot \cite{conversationBot} ein Array aus Konstanten. So lässt sich die Reihenfolge der States auch ganz leicht ändern. Soll der Bot bspw. E\-Mail und Telefonnummer am Afang abfragen oder sollen einige Schritte aus dem Konversationsfluss genommen werden, so sind diese hier einfach zu entfernen und die Nachrichten in den eizelnen Stufen leicht anzupassen.

            \paragraph{MESSAGES}
                Um Nachrichten an den Nutzer zentralisiert zu verwalten, verweisen Handler-Methoden wo immer möglich auf eine Konstante aus dem \emph{MESSAGES} Dictionary. So wird vermieden, dass Strings bei Anpassungen der Zustände oder deren Reihenfolge in mehreren Dateien angepasst werden müssen.
            
            \paragraph{KEYBOARD\_MARKUPS}
                Gleiches gilt für individuelle Tastaturen. 
                

        \subsection{status.py}
            \paragraph{Methode status}
                Zu jedem Zeitpunkt, kann der Nutzer seinen aktuellen Status abfragen. Dazu prüft die Methode status zunächst, ob der Nutzer überhaupt in der Datenbank existiert. Hat der Nutzer seine Informationen nämlich gelöscht, existiert er für den Bot nicht. Zwei Szenarien: 
                \begin{enumerate}
                    \item Der Bot findet den Nutzer, gibt den aktuellen Status zurück und beendet die Konversation.
                    \item Der Bot findet den Nutzer nicht und zeigt dem Nutzer Optionen an, fortzufahren - namentlich die Hilfe aufzurufen oder eine neue Konversation mit dem Bot zu starten.
                \end{enumerate}
        

        \subsection{validation.py}
            Alle Input\-Validation Methoden sind ähnlich mit einem try/except oder if/else Block aufgebaut.

            \paragraph{validate\_birthdate}
                Die Methode bekommt die Nutzereingabe übergeben und vergleicht diese via der Methode strptime aus der datetime\-Bibliothek \cite{datetime} mit dem in der DACH-Region gängigen Datums\-Format: \emph{TT.MM.JJJJ}
                Stimmt die Eingabe mit dem definierten Format überein, gibt die Methode True zurück. Ansonsten wird ein ValueError geloggt und die Methode gibt False zurück.

            \paragraph{validate\_email}
                Diese Methode bedient sich eines relativ einfach regulären Ausdrucks: \verb/[A-Za-z0-9._%+-]+@[A-Za-z0-9.-]+\.[A-Z|a-z]{2,}/, um zu prüfen, ob die Eingabe eine E\-Mail sein könnte. (Ein vollumfänglicher regulärer Ausdruck, ein externder Dienst oder gar das versenden einer Test\-E\-Mail wurden aus Performance\-Gründen ausgeschlossen.)
                Ist der Vergleich erfolgreich, gibt die Methode True zurück, ansonsten False.

            \paragraph{validate\_telephone}
                Auch Telefonnummern werden via regulärem Ausdruck geprüft: \verb/^\+4[139]\d{9,12}$/
                Zugelassen sind so alle Telefonnummern aus Deutschland, Österreich und der Schweiz.
                Ist der Vergleich erfolgreich, gibt die Methode True zurück, ansonsten False.

        
        \subsection{cancel}
            \paragraph{Methode cancel}
                Wurde eine Konversation mit dem Main\-ConversationHandler gestartet, so kann diese auch manuell wieder beendet werden. So hat ein Nutzer, wann immer er sich im Konversationsfluss befindet, die Möglichkeit den Befehl /cancel abzusetzen. Da dieser Befehl im Main\-CommandHandler als Fallback definiert ist, kann der Befehl nur abgesetzt werden, solange dieser aktiv ist. Wird die Methode cancel aufgerufen, so wird ein Log-Eintrag über den Abbruch der Konversation abgesetzt. Direkt darauf werden alle Daten des Nutzers aus der Datenbank gelöscht und eine Bestätigung an den Nutzer ausgegeben. Sollte es bei diesem Vorgang zu einem Fehler kommen, so wird der Nutzer auch darüber benachrichtigt und es werden sowohl der Fehler, als auch ein Log-Eintrag in der Konsole ausgegeben. Diese Ausnahme tritt meist dann auf, wenn der Nutzer seine Daten bereits gelöscht und den Bot noch nicht neu gestartet hat - es ihn also in der Datenbank gar nicht gibt oder (selten), falls die SQL-Operation nicht erfolgreich war.
                Schließlich beendet der Bot die Konversation.

            \paragraph{Methode delte}
                Grundsätzlich handelt es sich bei der Methode delete um fast den gleichen Funktionsumfang, wie bei der Methode cancel. Allerdings ist sie nicht Bestandteil des Main\-ConversationHandlers, sondern in ihrem eigenen Handler definiert und kann somit zu jederzeit über den Befehl /delte aufgerufen werden. So ist dafür gesorgt, dass der Nutzer seine Daten auch löschen kann, wenn die Konversation mit dem Bot aus irgendeinem Grund unterbrochen oder bereits beendet wurde.


    \section{Datenbank}

        In diesem Abschnitt wird die Funktionsweise des Database\-Connectors erleutert. Alle Methoden sind ähnlich aufgebaut. Zunächst wird eine Verbindung zur Datenbank geöffnet, es finden diverse Prüfungen statt, eine CRUD-Operaion wird abgesetzt und die Antwort entweder innerhalb der Methode analysiert und ein Boolean oder der Wert aufbereitet und im entsprechenden Format zurückgegeben.
                
        \subsection{create\_db.py}
            \paragraph{Methode create\_db}
            Es wird versucht, eine Verbindung zur Datenbank (db) aufzubauen. Ist dies erfolgreich, wird der cursor erstellt, über den alle folgenden Operaionen an die Datenbank kommuniziert werden. Ist dies nicht erfolgreich, wird eine neue Datenbank, mit dem vordefinierten Namen erstellt. 
            Ist die Datenbank verfügbar und wurde eine Verbindung aufgebaut, so wird zunächst geprüft, ob es die Tabelle users schon gibt. Ist dem so, wird die Methode mit einem Commit und dem Schließen der Verbindung zur Datenbank, beendet. Ist dem nicht so, wird die Tabelle für die Benutzer instanziiert. Aufgrund der Simplizität der gespeicherten Daten kommt der Bot mit einer Tabelle aus.


        \subsection{select\_db.py}

            \paragraph{Methode user\_search}
                An user\-search bekommt eine Nutzer\-ID übergeben und prüft, ob ein Nutzer in der Datenbank existiert. Falls ja, wird True zurückgegeben - ansonsten False.

            \paragraph{Methode get\_all\_data}
                An get\_all\_data wird ebenfalls eine Nutzer\-ID übergeben und direkt eine Abfrage für alle Informationen abgesetzt, die es über diesen Nutzer gibt. Die Methode iteriert über alle Einträge, die gefunden wurden und gibt die Daten als Liste und in der Konsole zurück.
                
            \paragraph{Methode get\_customers}
                Die Methode hat keine Input-Parameter, sondern gibt die gesamte Nutzertabelle aus und diese als Liste zurück.

            \paragraph{get\_value}
                An get\_value werden eine Nutzer\-ID sowie eine Spaltenbezeichnung übergeben. So kann ein spezifischer Wert aus der Datenbank abgerufen werden. 

        \subsection{insert\_value\_db.py}
            \paragraph{Methode insert\_update}
                Um sicherzustellen, dass eine Datebank existiert, bevor man in sie hineinschreibt, wird zunächste die Methode create\_db() aufgerufen. Diese kommt schnell zurück, da die Datenbank in den meisten Fällen bereits existiert. Nun wird in einem try/except\-Block versucht, einen existierenden Datenbankeintrag zu aktualisieren. Das funktioniert meistens, weil der Datensatz für einen Nutzer bereits bei der Eingabe von /start passiert. So kann ein Eintrag bereits von Beginn an immer weiter angereichert werden. Nachdem der Eintrag erfolgreich aktualisiert wurde, gibt es noch einen Log\-Eintrag auf der Konsole.  
        
        \subsection{insert\_update\_db.py}
            \paragraph{Methode insert\_update}
                Die Methode funktioniert ähnlich wie die Methode insert\_update aus der insert\_value\_db.py. Sie unterscheidet sich darin, dass sie alle Parameter für den gesamten Datenbankeintrag eines Nutzers übergeben bekommt. So ist es möglich, einen Nutzer auf einmal in die Datenbank einzufügen, ohne den Bot jedes Mal zu durchlaufen. Vor allem zu Testzwecken ist diese Methode hilfreich.
        
        \subsection{delete\_record.py}
            \paragraph{Methode delete\_record}
                Die Methode bekommt eine Nutzer-ID übergeben und prüft zunächst via der Methode user\_search(), ob es den Nutzer mit der angegebenen Nutzer-ID überhaupt gibt. Falls ja, wird in einem try/except\-Block versucht, alle Informationen eines Nutzers via SQL-Befehl zu löschen. 


            \paragraph{Methode delete\_value}
                Die Methode ist mit der Methode delete\_record fast identisch. Hier wird zusätzlich eine Spaltenbezeichnung übergeben, die es ermöglicht, einen einzelnen Wert zu löschen.



    \section{E\-Mail\-Versand}



    
    
    \section{}
    
    \section{}
    
    \section{}
    
    \section{}
    
    \section{}
    
    
    
    \section{Kalender}
    Google Cloud Console: https://console.cloud.google.com/apis/credentials/consent?project=coaching-bot-339115 
    Calendar API Python Quickstart Guide: https://developers.google.com/calendar/api/quickstart/python 
    Verify own website: https://www.google.com/webmasters/verification/home?hl=en
    
    
    \paragraph{Tests}
    Das Haupt\-Test\-Instrument ist die Telegram\-App selbst. Daneben wurden 2 Testskripte zum einfachen Testen verschiedener Teile der App geschrieben. Tests sind getrennt und entsprechend ein\- oder auskommentiert. 