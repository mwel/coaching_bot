\chapter{Konzept} \label{Konzept}


Um sich auf die in Kapitel \ref{Einleitung und Problemstellung} Einleitung und Problemstellung genannten Zielgruppen (Personas Nutzer und Coach) einzulassen, hat man sich nach einiger Analyse entschieden, in einem ersten Schritt einen Chat-Bot zu programmieren, der basale Informationen vom Nutzer abfragt und den Nutzer einen Termin vereinbaren lässt. Weitere Iterationen sind nach erfolgreicher Beta-Phase möglich und ein Ausblick wird in Kapitel \ref{Zusammenfassung und Ausblick} Zusammenfassung und Ausblick gegeben. \\ Im Folgenden wird zunächst eine User Story, dann ein darauf zugeschnittenes Lösungskonzept erklärt. Danach wird anhand eines Schaubilds aufgezeigt, welche Systeme für den Bot und dessen Funktionalität relevant sein werden und wie diese zusammenarbeiten sollen. 

\section{User Journey und Features} \label{Konzept: User Journey}
	Der Nutzer soll einen vordefinierten Workflow durchlaufen, der im Folgenden skizziert und in Kapitel \ref*{Realisierung} Realisierung näher beschrieben wird:

	\begin{enumerate}
		\item Ein Nutzer kommt i.e. via Link oder QR-Code zum Coaching Bot.\footnote{\url{https://t.me/thecoachingbot?start=start}}
		\item In Telegram angekommen öffnet sich ein neuer Chat mit dem Coaching Bot.
		\item Der Bot stellt sich vor und beginnt eine Reihe an Fragen zu stellen. Antworten oder deren Format sind z.T. vordefiniert oder werden vorgeschlagen.
		\item Der Nutzer teilt Texte, Bilder und andere Informationen.
        \item Sowohl Coach als auch Coachee sollen spätestens nach Abschluss des Konversationsflusses die Möglichkeit haben, eingegebene Daten einzusehen. Der Nutzer erhält zu diesem Zweck automatisch eine Zusammenfassung via Chat und per E-Mail und kann sie zusätzlich manuell vom Bot abfragen. Dem Coach werden alle Informationen in einer einfachen Übersicht über alle Nutzer dargestellt.
        \item Sobald alle Informationen eingereicht sind, erhält der Nutzer die Möglichkeit, einen Termin zu vereinbaren. Dem Nutzer werden dazu drei Terminvorschläge über die nächsten zehn Tage angeboten, aus denen er frei wählen kann. Die Dauer pro Termin beträgt 50 Minuten. Es können nur Termine ausgewählt werden, die noch nicht belegt sind und innerhalb der Geschäftszeiten liegen. (Auf die Auswahl der Zeitzone wird hier verzichtet, da das Coaching-Angebot aktuell nur in mitteleuropäischer Zeit angeboten wird.) Im Hintergrund wird ein Google Calendar gemanaged.
        \item Der Nutzer bekommt eine schriftliche Bestätigung mit dem vereinbarten Termin.\footnote{Sofern der Nutzer seinen Kalender so konfiguriert hat.} Darüber hinaus kann der Termin jederzeit vom Bot abgefragt werden.
        \item Der Bot verabschiedet sich. (Ende der Konversation)

    \end{enumerate}
	Der Prozess kann natürlich zu jeder Zeit unter- oder abgebrochen werden. Außerdem besteht die Möglichkeit für den Nutzer, seine Daten jederzeit einzusehen, zu löschen oder den Prozess neu zu starten.

	Für den Coach sollen eingereichte Informationen über alle Nutzer inkl. Termin in einer Übersicht im Web-Browser eingesehen werden können. Außerdem sind alle vereinbarten Termine auch im Calender des Coaches ersichtlich.

\section{Technischer Aufbau}

	\begin{figure} %[hbtp]
		\centering
		\includegraphics[width=1.0\textwidth]{images/220320_PA28464_Architecture.png}
		\caption{Konzeptionelle Architektur für das Projekt \emph{Der Coaching Bot}}
		\label{fig: architecture}
	\end{figure}

	Wie in Abbildung \ref{fig: architecture} rechts mittig skizziert, besteht der Kern des Bots aus einem endlichen Automaten (State Machine), der Zustände vordefiniert und festlegt, wann sich welcher Nutzer in welchem Zustand befindet und von welchem in welchen Zustand er sich wann bewegen darf. An diesen Kern sind als zentrales Steuerungselement des Bots alle anderen Systeme angebunden. Dazu gehören:
	\begin{enumerate}
		\item Die SQLite Datenbank zur Speicherung der Nutzerdaten
		\item Die Telegram API, über die die Kommunikation mit dem Telegram Client abgewickelt wird
		\item Die Google Calendar API, über die Events erstellt und versendet werden können
		\item Der Mail Server, über den E-Mails an den Nutzer versendet werden können.
	\end{enumerate} 
 
	Der Nutzer interagiert mit der Applikation durch drei Kanäle - in Abb. \ref{fig: architecture} links:
	\begin{enumerate}
		\item Telegram Client: Kommunikation mit dem Bot
		\item Calendar Client: Erhalt/Annahme/Ablehnung der vereinbarten Termine
		\item Mail Client: Erhalten der Zusammenfassung und Bestätigung
	\end{enumerate}
	
	Der Bot wird von Nutzern (in Abb. \ref{fig: architecture} links oben: User Persona Coachee) via einem der verfügbaren Telegram Clients (Mobile oder Desktop) angesprochen und reagiert auf die Eingabe entsprechend. So können verschieden Funktionen ausgelöst werden. Bspw. werden Antworten zurückgegeben, Informationen gespeichert oder es wird ein Vorschlag gemacht und an den Nutzer zurückgegeben. Der Bot soll mit mehreren Benutzern gleichzeitig sprechen können. Das wird ermöglicht, weil alle Reaktionen des Bots mit der Kennung des jeweiligen Nutzers verknüpft sind. So spricht der Bot den Nutzer mit Namen an oder kann sich daran erinnern, welche Fragen schon beantwortet wurden und welche nicht. \\ \\
	
	Der Coach interagiert mit der Applikation nur durch einen Kanal (in Abb. \ref{fig: architecture} links unten) - den Web Browser. Hier steht eine einfache Übersicht über Anmeldungen, gesammelte Informationen und der aktuelle, monatliche Terminkalender zur Verfügung.\footnote{Der Kalender könnte natürlich auch über einen Calender-Client synchronisiert werden.}

\section{Zustände \& Konversationsfluss}

	Im folgenden Zustandsdiagramm (Abb. \ref*{fig: state machine}) ist der Konversationsfluss des Bots auf hohem Abstraktionsniveau - als endlicher Automat (State Machine) - abgebildet. Es wird deutlich, dass der Bot einem Skript folgt, das den Konversationsfluss repräsentieren soll und Loops soweit als möglich vermieden werden. Der Haupt-Pfad ist fett eingezeichnet. Sobald der Bot gestartet wird, befindet er sich im Zustand \verb|S0| (START). Der Nutzer löst Übergänge durch seine Eingabe aus und wird durch die State Machine automatisch zum entsprechenden Zustand geleitet. Daneben besteht für die meisten Schritte die Möglichkeit für den Nutzer, einen Zustand zu überspringen, wenn er eine Angabe nicht machen möchte (hier als dünne durchgezogene Linie erkennbar). \\
	Was aber, wenn der Nutzer den Vorgang unterbrechen möchte oder der Bot aufgrund eines technischen Grundes neu gestartet werden muss? Es muss dem Nutzer möglich sein, nach einiger Zeit zum Bot zurückzukehren und an der Stelle weiterzumachen, an der er aufgehört hat. Zu diesem Zweck und als zentraler Einstiegspunkt dient der Zustand \verb|S0| (START). Hier wird analysiert, ob der Nutzer schon bekannt ist und falls ja, bis wohin er den Prozess bereits durchlaufen hat. Dann wird er dorthin weitergeleitet. Daher ist es möglich von START aus zu allen anderen Zuständen zu gelangen, auch wenn das nicht die Regel ist. (Weiterleitungen wie \verb|return| oder \verb|cancel| hier gestrichelt eingezeichnet) Hat der Nutzer den Prozess bereits abgeschlossen, so kann er sogar von \verb|S0| (START) direkt in S10 (ENDE) landen und wird entsprechend benachrichtigt. Da dem Nutzer die Möglichkeit gegeben werden soll, den Prozess jederzeit zu beenden, ist es auch möglich, von jedem Zustand in \verb|S10| (ENDE) zu gelangen. Die Listenansicht aus Tabelle 4.1 in Kombination mit der Grafik in Abb. \ref*{fig: state machine} illustrieren, wo die Konversation beginnt und welche Zustände und Übergänge möglich sind. \footnote{Darüber hinaus ist der Konversationsfluss in einer detailreicheren Ansicht verfügbar: \url{https://github.com/mwel/coaching_bot/blob/main/thesis/images/220320_PA28464_Conversation_Flow.svg}}
	
	\label{states table}
	\begin{table} %[hbtp]
		\centering
		\begin{tabular}{l | l l l l}
			\textbf{Zustände} 	&		\textbf{Bezeichnung}	&		\textbf{Mögliche Übergänge zu}\\
			\hline
			S0 					&		START 					&		alle Zustände\\
			S1 					&		BIO 					&		GENDER, BIRHTDATE, END\\
			S2 					&		GENDER 					&		BIRHTDATE, EMAIL, END\\
			S3 					&		BIRTHDATE 				&		EMAIL, END\\
			S4 					&		EMAIL 					&		TELEPHONE, END\\
			S5 					&		TELEPHONE 				&		LOCATION, PHOTO, END\\
			S6 					&		LOCATION 				&		PHOTO, SUMMARY, END\\
			S7 					&		PHOTO 					&		SUMMARY, END\\
			S8 					&		SUMMARY 				&		APPOINTMENT, ENDE\\
			S9 					&		APPOINTMENT				&		END\\
			S10 				&		END 					&		Ende: Applikation beendet (Neustart möglich)\\
			
		\end{tabular}
		\caption{Zustände und Übergänge des Konversationsflusses}
		\label{tab: states}
	\end{table}
	
	\begin{figure} %[hbtp]
		\centering
		\includegraphics[width=1.0\textwidth]{images/220328_PA28464_State-Machine.png}
		\caption{Endlicher Automat / State Machine des Konversationsflusses des Bots}
		\label{fig: state machine}
	\end{figure}
	