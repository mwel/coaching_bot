\label{Grundlagen}
\chapter{Grundlagen}
    Die folgenden Sprachen und Systeme dienen als Grundlage für die im Rahmen dieses Projekts entwickelte Applikation. Eine Liste aller eingebundenen Bibliotheken kann dem Pipfile des Projets entnommen werden. 

    \section{Python}
        Der größte Teil der Applikation ist in Python 3.8.6 \cite{python3.8.6} geschrieben. Das war zum Stand des Entwicklungsbeginns 2021 die aktuell stabile Python-Version.

    \section{Telegram API}
        Die Applikation basiert auf der API des Instant Messaging Dienstes Telegram \cite{telegramAPI} sowie deren Extension \cite{telegramAPIext}.
        Als Grundgerüst der für den Bot erforderlichen State Machine wurde der ConversationBot von Leandro Toledo et. al. genutzt. \cite{conversationBot} Das Repository enthält eine Vielzahl basaler Bot-Implementierungen, die als Startpunkt für jegliche Bot-Implementierung einen guten Überblick über das Grundgerüst und die Funktionsweise eines Bot geben.

    \section{Telegram Chat Bots}
        Telegram Chat Bots sind Applikationen, die auf einer quelloffenen API \cite{telegramAPI} basieren, auf allen Komplexitätsstufen adaptierbar sind und es jedermann ermöglichen, einen Chatbot zu bauen. Die einzige suboptimale Einschränkung besteht im Vendor Lock-In der Telegram-App. (Der Bot ist nur in Verbindung mit der Telegram-App \cite{telegram} nutzbar.) Aufgrund der technischen Vorteile, die dieses Framework bietet, ist dieser Nachteil jedoch in Kauf zu nehmen. Vor Allem im Hinblick auf einen Proof of Concept, ist ein Bot auf einer Plattform hinreichend. Sollte das Prinzip Erfolg versprechen, so kann die Logik auf andere Umgebungen erweitert werden. \\
        Die meisten Menschen in Deutschland und der DACH-Region verwenden immer noch WhatsApp \cite{Nutzerzahlen} und doch bietet uns der populärste Messenger nicht die Freiheiten und den Funktionsumfang, den wir uns für unseren CoachingBot wünschen. Telegram jedoch hält genau diese Offenheit für uns bereit. \cite{telegramVergleich}. So bietet der Dienst, die Möglichkeit, via einer API direkt in die Entwicklung einzusteigen und hält sogar basale State-Mashines für uns bereit, die uns die Komplexität für den Kern des Bots nicht komplett abnehmen, aber als Gerüst für einen ChatBot dienen können.

    \section{SQLite}
        Zur Speicherung von Nutzerdaten wird eine SQLite Datenbank \cite{sqlite} genutzt.

    \section{SQLite3 API}
        Als Schnittstelle zwischen dem Python basierten Backend und der SQLite Datenbank nutzt die Applikation den sqlite3 Database-Connector \cite{sqlite3API}.

    % \section{Mailing}
        % Die Applikation versendet als Zusammenfassung aller eingereichten Informationen eine E-Mail von einem Mail-Server. 

    \section{HMTL}
        HTML bietet uns die Möglichkeit die grafische Oberfläche in einem Standard Webbrowser auszugeben. \cite{HTML}

    \section{CSS}
        Zur Aufbereitung der Web-GUI wird ein CSS-File genutzt.

    \section{Flask}
        Die HTML-GUI wird via Flask \cite{flask} an einen lokalen Web-Server übermittelt und kann so einfach mittels Python in gängigen Browsern präsentiert werden.


    \section{Google Calendar API}
        Die Applikation bindet die Google Calendar API \cite{googleCalAPI} an, um es dem User zu ermöglichen, einen Termin mit dem Anbieter zu vereinbaren.


    \section{TheCoachingBot}
        Schließlich findet sich der gesamte Source-Code inklusive aller Abhängigkeiten in einem öffentlichen GitHub Repository: \cite{repo} 