\label{Anwendungsszenarien}
\chapter{Anwendungsszenarien}

Der Coaching Bot kann in allerlei Szenarien angewandt werden. Die in den letzten Jahren stark angewachsene Zahl an Personal Coaches kann den Bot mit basalen Programmierfähigkeiten an die eigenen Bedürfnisse anpassen und ist flexibel in der Wahl der unterliegenden Infrastruktur. Vor allem für Nebenerwerbstätige Coaches mit einem kleinen Kundenportfolio, bei dem sich Kaltaquise oft als teuer und wenig erfolgreich herausstellt, bietet der Coaching Bot eine einfache Möglichkeit, Neukunden kostenfrei und einfach onzuboarden. Der Prozess ist unkompliziert, unverbindlich, gut dokumentiert und einfach zu adaptieren. 

\paragraph{Potenzial}
Sollte der Coaching Bot vielversprechende Ergebnisse liefern, sind folgende Ausbaustufen möglich:

\begin{enumerate}
    \item Verteilung und Verlagerung in die Cloud für konstante und hohe Verfügbarkeit
    \item Skripten und Automatisierung weiterer Coaching-Stufen. i.e. könnte die erste Session, die oft ähnlich abläuft auch vom Bot abgehandelt werden.
    \item Ton-Aufnahmen für das Biography-Modul
    \item Verteilung auf weitere Messenger-Dienste via \ref{BotMan} BotMan
    \item NLP Integration
    \item 
\end{enumerate}

