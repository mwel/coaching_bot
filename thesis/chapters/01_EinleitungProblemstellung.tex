\chapter{Einleitung und Problemstellung} \label{Einleitung und Problemstellung}
        
    Viele junge Menschen die nach einem abgeschlossenen Studium in die Arbeitswelt einsteigen möchten, haben keine oder wenig Erfahrung damit, wie sie sich vorbereiten sollen oder welche Schritte erforderlich sind, um einen erfolgreichen Start zu schaffen. Persönliche Beratungsleistungen und speziell Einzel\-Coaching können dabei helfen, sich effektiv vorzubereiten und einen erheblichen Wettbewerbsvorteil bieten, sind aber für Berufseinsteiger ohne signifikante finanzielle Mittel meist weder zugänglich noch erschwinglich. Aber auch Professionals mit hinreichenden Mitteln, haben oft Hemmungen und die Einstiegshürde ich hoch. Diese Hürde soll zum einen herabgesetzt werden. Um diese Zielgruppen unkompliziert und intuitiv anzusprechen und so die Einstiegshürde für derlei Services herabzusetzen, gewinnen digitale, mobile Messenger-Dienste und Social-Media-Kanäle zunehmend an Relevanz. Interaktionen mit jungen Absolventen auf klassischen Websiten stellen sich erfahrungsgemäß vor Allem im persönlichen Networking als wenig erfolgreich heraus. 
    
    Darüber hinaus geht man als Coach mit einem Erstgespräch immer in eine relativ riskante Vorleistung, weil die erste Kennenlern-Session meist gratis angenoten werden muss, um überhaupt Neukunden zu gewinnen. 
    Aus dem Bedürfnis, mit geringem kontinuierlichen Planungs- und Koordinationsaufwand ein breit gefächertes Klientel im Coaching-Bereich aufzubauen, hat man sich dazu entschieden, diverse Kanäle zu prüfen und ggf. Systeme und Technologien für einen standardisierten Onboarding-Mechanismus und entsprechenden Workflow zu etablieren, um den Prozess bis zum Kennenlernen von manuellem Aufwand soweit als sinnvoll zu abstrahieren und Kosten dahingehend auf ein Minimum zu reduzieren. Der Standardisierungscharakter ist deshalb sinnvoll, weil eine erste Kontaktaufnahme und Vorbereitung auf eine erste Sitzung erfahrungsgemäß sehr kongruent zueinander oder gar einem Skript folgend verlaufen. \\
    \\
    Die Abwendung vom Web-Browser als Kommunikationsmedium ist keine Neuerung der letzten Jahre und schreitet mit der steten Optimierung von Messenger-Diensten wie WhatsApp, Signal, Telegram und Kommunikationsoptionen via Social Media Portalen wie Instagram, Facebook, TikTok, SnapChat, etc. weiter voran. Ein Nebeneinander von Web-Formular und Chat-Bot kann helfen, die Conversion Rate für eine Dienstleistung auf ein neues Niveau zu heben. \cite{conversion} In diesem Projekt soll daher ein Proof of Concept für einen Chat-Bot entwickelt werden, der ein Web-Formular als weiteren Kanal ergänzt.\\ 
    \\ 
    \paragraph{Im Fokus stehen:}
    \begin{enumerate}
        \item Conversion und Hauptziel: Ein Nutzer kann einen Termin mit dem Dienstleister ohne dessen Zutun vereinbaren.
        \item Reduktion manueller Aufwand: Dem Anbieter der Coaching-Dienstleistungen soll die repetitive, manuelle Akquise durch einen geführten, technisch gestützen, Ablauf erleichtert werden. 
        \item Weiterer Akquise-Kanal: Es soll eine Alternative für den Kanal Web-Formular gefunden werden. 
        \item Bot-Charakter: Der weitere Kanal soll wie ein Chat-Bot wirken.
        \item Einfache Adaption: Es soll ein System gebaut werden, das durch Anbieter von Coaching-Dienstleistungen mit wenigen Vorkenntnissen leicht adaptierbar ist.
    \end{enumerate}
    
    Ziel des Projekts ist es, eine erste Version des Coaching Bots zu programmieren, die es ermöglicht, persönliche Angaben zu machen und den Nutzer zu einer Terminvereinbarung hinführt. Um den Entwicklungsaufwand so gering, wie möglich zu halten, wurde hierzu eine Reihe an Systemen analysiert. Eine detaillierte Aufschlüsselung der Systeme folgt in \ref{Verwandte Arbeiten} Verwandte Arbeiten und \ref{Grundlagen} Grundlagen.




