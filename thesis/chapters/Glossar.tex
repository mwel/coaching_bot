\chapter{Glossar}

\abbreviation{API}		        {Anwendungsprogrammierschnittstellen (APIs) vereinfachen Softwareentwicklung und -innovation, indem sie Anwendungen den einfachen und sicheren Austausch von Daten und Funktionen ermöglichen. \cite{apiIBM2020}}
\abbreviation{Bot}			    {hier: ChatBot - Computerprogramm, das mit einem Nutzer interagieren und vordefinierte Aufgaben selbstständig erledigen kann.}
\abbreviation{PoC}			    {Proof of Concept: z. dt. Machbarkeitsstudie}
\abbreviation{DACH}	            {Deutschland, Österreich, Schweiz bzw. deutschsprachige Region Europas}
\abbreviation{Vendor Lock-In}	{hier: Beschränkung der Anwendbarkeit eines Systems auf einen Anbieter - in diesem Fall 'Telegram'}
\abbreviation{GUI}			    {Graphical User Interface: z.dt. Benutzeroberfläche}
\abbreviation{CSS}              {Cascading Style Sheets: Stylesheet-Sprache, die zur optimsch ansehnlicheren Aufbereitung von Benutzeroberflächen genutzt wird}
\abbreviation{Geo-Fencing}      {Virtuelle Abgrenzung eines festgelegten Gebiets oder einer Region}
\abbreviation{CRUD}             {CREATE, READ, UPDATE, DELETE: Standard-Operationen, die auf einer Datenbank durchgeführt werden}
\abbreviation{SQL}              {Search Query Language:  Sprache zur Definition von Datenstrukturen in relationalen Datenbanken}

