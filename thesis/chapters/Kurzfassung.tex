\kurzfassung

Anmeldung und Terminvereinbarung für Coaching-Programme sollen automatisiert werden. Gängig wird dies über ein Webformular abgewickelt. Allerdings erfreut sich diese Herangehensweise keiner großen Beliebtheit und Conversion Rates sind niedrig. Daher wird ein weiterer Kommunikationskanal gesucht, der dabei helfen die Erfolgschancen zu erhöhen. Diese Arbeit behandelt den Versuch, eine Chat-Bot-Technologie zu nutzen, um die Problematik zu lösen. Nach einer etwas detaillierteren Einführung in die Problematik, wird eine Reihe an bereits verfügbaren Systemen analysiert. Eine Analyse dieser Systeme gegen Ausschlusskriterien zeigt, dass Viele für die dieses Projekt nicht geeignet sind. Daher wird aus einer Kombination und Erweiterung mehreren Systeme ein Lösungskonzept erarbeitet und präsentiert. Die Realisierung erläutert technisch abstrakt, wie die Systeme angepasst und miteinander verknüpft werden, bevor im Kapitel Implementierung detailliert erklärt wird, wie genau das System funktioniert und relevante Ausschnitte aus dem Programmcode in Abschnitten vorgestellt und erklärt. Dabei werden einige Teile der Applikation stärker beleuchtet als Andere. Auf zentrale Funktionen und komplexere Logiken und Systemzusammenhänge wird dabei genauer eingegangen. Da der Bot von Interessierten Entwicklern mit wenigen Vorkenntnissen genutzt und weiterentwickelt werden soll, wird versucht, ein Verständnis für diese Zielgruppe herzustellen, das hinreichend ist, um das System betreiben und anpassen zu können. \\
Nach einer Führung durch die laufende Applikation aus Nutzersicht und einigen visuellen Beispielen für die Anwendung und Interaktion wird noch auf potenzielle Anwendungsszenarien eingangen, die neben dem Haupt-Anwendungsfall weiteres Potenzial der Applikation erahnen lassen. \\
Die Arbei schließt mit einer Zusammenfassung und einem Ausblick für weitere, mögliche Entwicklungsstufen und Verbesserungsvorschläge für die Applikation. 

\kurzfassungEN

Registration and managing appointments for coaching programs should be automated. This is usually done via a web form. However, this approach is rather unsuccessful, which is why an alternative is to be found. This work makes the attempt to use a chat bot technology to solve the problem. After a more detailed introduction to the problem, a number of already available systems are analyzed. It is quickly revealed that these do not meet the requirements for a variety of reasons. Therefore, a solution concept is developed and presented from a combination and extension of several existing systems, followed by a detailed description of the realization and implementation of that concept. Parts of the application are exemplarily highlighted more than others. Challenges and more complex system and their relationships are addressed in particular. It is explained in detail how the coaching bot works in order to establish an understanding that is sufficient for other coaches to be able to adapt the system even with little previous programming knowledge. \\
After some visual examples of how to use and interact with the new system, potential application scenarios are discussed, followed by a brief summary and outlook for further development. 

Registration and setting up appointments for coaching programs should be automated. This is usually done via a web form. However, this approach is rather unsuccessful and conversion rates are low. Therefore, an additional communication channel is to be found to help increase chances of success. This paper deals with the attempt to use a chat bot technology to solve the problem. After a more detailed introduction to the problem, a number of already available systems are analyzed. An analysis of these systems against exclusion criteria shows that many are not suitable for this project. Therefore a solution concept is developed and presented from a combination and extension of several systems. The realization explains on a  technically abstract level, how the systems are adapted and linked together, before a detailed explanaition of how exactly the system works and relevant excerpts from the program code are presented and explained in around selcted snippets. Some parts of the application are highlighted more than others. Central functions and more complex logics and system interrelationships are dealt with in more detail. Since the bot is intended to be used and further developed by interested developers with little previous knowledge, an attempt is made to create an understanding that is sufficient to enable this group to operate and adapt the system. \\
After a guided tour through the running application from the user's point of view and some visual examples of the application and interaction, potential use cases and scenarios will be discussed, which give an idea of further potential for the application besides the main use case. \\
The paper concludes with a summary and an outlook for further, possible development stages and suggestions for improvement of the application.