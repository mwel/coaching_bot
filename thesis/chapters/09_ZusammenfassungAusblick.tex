\label{Zusammenfassung und Ausblick}
\chapter{Zusammenfassung und Ausblick}

    Ziel des Projekts war es, dem interessierten Nutzer eine Alternative zum klassischen Webformular anzubieten, über die er sich für einen Coaching-Service anmelden kann. Er soll die Möglichkeit erhalten, einige Informationen über sich zu vermitteln, diese sollen einem Coach in einer einfach Übersicht übermittelt werden können. Schließlich soll es zu einem Treffen zwischen Coach und Coachee kommen. Dazu muss ein Termin vereinbart werden können. \\
    All diese Dinge bietet uns der Coaching Bot und dies unter den Randbedingungen, dass er quelloffen, kostenlos und einfach adaptierbar ist. Insofern wurden die wichtigsten Kriterien an das Projektgewerk erfüllt. Allerdings bleibt der Bot ein Proof-of-Concept und birgt natürlich Weiterentwicklungspotenzial. Coaches mit ähnlichen Bedürfnissen können den Bot selbst nutzen und hosten. Aufgrund der vielen, mächtigen, massentauglichen Systeme, die es auf dem Markt gibt und der niedrigen Priorität, die das Thema Datensicherheit gegenüber internationalen Korporationen für für Viele immer noch hat, wird die Mehrheit aber wahrscheinlich eher zu plug-and-play-Produkten i.e. WhatsApp oder Facebook Messenger von Meta greifen, die sich immer weiter in Richtung Bots entwickeln. \\

    Für jene aber, die einen einfach adaptierbaren Bot für ihr Onboarding suchen und die sich und ihre Kunden gerne gegen die omnipräsente Datensammelfreude der Konzerne schützen möchten, stellt der Coaching Bot eine solide Startplafform dar. \\ \\

    So sind bspw. folgende Ausbaustufen möglich:

    \begin{enumerate}
        \item Verteilung der Systeme und Verlagerung in die Cloud für konstante und hohe Verfügbarkeit. (Containerisierung wäre i.e. eine Option.)
        \item Skripten und Automatisierung weiterer Coaching-Stufen. i.e. könnte die erste Session, die oft ähnlich abläuft auch vom Bot abgehandelt werden.
        \item Ton-Aufnahmen für das Biography-Modul
        \item Verteilung auf weitere Messenger-Dienste, i.e. via \ref{BotMan} BotMan
        \item NLP- oder Sprachkatalog-Integration (So würde die Sprache des Bots etwas variieren und natürlicher wirken.)
        \item Ausbau der Web-GUI. Hinsichtlich der darzustellenden Informationen sowie zum Thema Daten-Sicherheit besteht hier Potenzial.
    \end{enumerate}
