\label{Zusammenfassung und Ausblick}
\chapter{Zusammenfassung und Ausblick}

    Ziel des Projekts war es, dem Nutzer eine Alternative zum klassischen Webformular anzubieten, über die er sich für einen Coaching-Service anmelden kann ohne, dass dafür eine Interation (Telefongespräch, E-Mail-Verkehr) zwischen Coachee und Coach erforderlich sind. Der Nutzer soll die Möglichkeit erhalten, einige Informationen über sich in einem Konversations-Charakter zu vermitteln. Diese sollen einem Coach in einer einfachen Übersicht angezeigt werden können. Schließlich soll es zu einem Treffen zwischen Coach und Coachee kommen. Dazu muss ein Termin vereinbart werden können. \\
    All diese Features bietet uns der Coaching Bot und dies unter den Randbedingungen, dass er quelloffen, kostenlos, auf allen gängigen Endgeräten verfügbar und einfach adaptierbar ist. Insofern konnten die in Abschnitt \ref*{Einleitung und Problemstellung} definierten Kriterien an das Produkt erfüllt werden. \\
    
    Zum Zeitpunkt des Abschlusses dieser Projektarbeit\footnote{siehe Repository-Status vom 28.03.2022: \url{https://github.com/mwel/coaching_bot/tree/6cc113fc1ceeff5bb34b0ff99b81049496f79568}} ist der Bot ein Proof-of-Concept. Es konnte gezeigt werden, dass es möglich ist, die komplette User Story für den Coachee in Python mithilfe der Telegram- und der Google-Calendar-API zu realisieren. Der Conversation Bot wurde so adaptiert und erweitert, dass er den Konversationsfluss der User Story aus Kapitel \ref*{Konzept} Konzept von Anfang bis Ende abdecken kann und das ohne Interaktion des Coaches. Angaben des Nutzers können persistent gespeichert und wieder gelöscht werden. Der Nutzer hat also jederzeit die volle Kontrolle über seine Daten. Schließlich ist das System einfach zu installieren und selbst für Neulinge der Programmierung leicht adaptier- und erweiterbar.\\
    \\
    In einer nächsten Phase muss das System mit echten Nutzern getestet, peer reviewt und weiterentwickelt werden, um eine optimale User Experience bieten zu können. Davon abgesehen bestehen aktuell bereits folgende Weiterentwicklungsbestrebungen: 

    \begin{enumerate}
        \item Verteilung der Systeme und Verlagerung in die Cloud für konstante und hohe Verfügbarkeit. Aktuell wird das System lokal betrieben. Der Mail-Server und die beiden APIs befinden sich natürlich bereits in der Cloud, aber für eine konstante und hohe Verfügbarkeit sollten auch der Bot selbst und die Flask-App entweder auf einen Container-Service, eine VM in einem Datacenter oder in eine klassische Web-Site migriert werden. Datenschutztechnisch ist hier pro System zu prüfen, welchen Regularien es dann unterliegt. 
        \item Skripten und Automatisierung weiterer Coaching-Stufen. Aktuell führt der Konversationsfluss den Nutzer vom Einstieg bis zur Terminvereinbarung. In Zukunft könnten auch weitere User-Stories abgebildet werden. So wäre es bspw. denkbar, dass der Bot den Coach in einer erste Session ersetzt und Fragen, die meistens in einer ersten Sitzung gestellt werden, abfragt.
        \item Ton-Aufnahmen anstatt freier Texteingabe. Telegram bietet ein Ton-Aufnahme-Modul, durch das Audio-Nachrichten versendet und empfangen werden können. Anstatt den Nutzer nach einer textlichen Eingabe zu bitten, wäre es interessant zu testen, ob Nutzer nicht lieber eine Sprachnachricht schicken. Diese kann dann zu einem beliebigen Zeitpunkt bis zur ersten Sitzung vom entsprechenden Coach angehört werden.
        \item Skalierung auf weitere Messenger-Dienste. Aktuell existiert der Bot nur im Telegram-Universum. Via einer Lösung wie Bottender, Claudia Bot Builder oder BotMan könnte die Reichweite des Bots auf weitere Messenger-Dienste ausgeweitet und erheblich erhöht werden.
        \item NLP- oder Sprachkatalog-Integration. Der Coaching Bot in seinem jetzigen Zustand nimmt Texte und Daten entsprechend einem einfachen Filter entgegen. Möchte man die natürliche Sprache des Nutzers interpretieren und den Bot befähigen, dynamisch und individuell darauf zu antworten, so wäre es interessant, Systeme mit NLP-Fokus näher zu betrachten und Integrationsmöglichkeiten zu prüfen. 
        \item Optische Optimierung und Ausbau der Web-GUI. Die Web-GUI in ihrem jetzigen Zustand bietet eine rein funktionale Ansicht der Nutzerinformationen für den Coach. Features wie eine direkte Chat-Möglichkeit mit dem Coachee, die Telefonnummer klickbar zu machen, Termine anpassen zu können, wären leicht umsetzbar. Besonders interessant wäre es für den Coach, wenn er direkt aus der Übersicht ein Profil des Coachees generieren und dann mit diesem interagieren könnte.
        
    \end{enumerate}

    Im Hinblick auf das Potenzial, das Chat-Bot-Technologien für die nächsten Jahre in der Kundenkommunikation bereithalten, erhoffen wir uns, dass der Coaching-Bot weiterentwickelt wird.