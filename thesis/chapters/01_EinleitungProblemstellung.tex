\chapter{Einleitung und Problemstellung}


    \section{Motivation}
        
        \paragraph{Warum einen Coaching Bot bauen?}

            Viele junge Menschen die nach einem abgeschlossenen Studium in die Arbeitswelt einsteigen möchten, haben keine oder wenig Ahnung davon, wie sie sich vorbereiten sollen oder welche Schritte erforderlich sind, um einen erfolgreichen Start zu schaffen. Persönliche Beratungsleistungen und speziell Einzel\-Coaching können dabei helfen, sich effektiv vorzubereiten und einen erheblichen Wettbewerbsvorteil bieten, sind aber für Berufseinsteiger ohne signifikante finanzielle Mittel miest weder zugänglich noch erschwinglich.
            Das sich in Gründung befindliche Unternehmen -wavehoover AG- aus Zürich hat es sich daher zur Aufgabe gemacht, in dieser Hinsicht einen Beitrag zu leisten. Interaktionen mit jungen Absolventen auf klassischen Websiten waren bis dato wenig erfolgversprechend.\\ 
            Aus diesem Bedürfnis heraus und aufgrund beruflicher Erkenntnis, dass die erste Kontaktaufnahme und die Vorbereitung auf die erste Sitzung meist sehr ähnlich oder gar einem Skript folgend ablaufen, ist die Idee zu einem standardisierten und automatisierten OnBoarding-Prozess entstanden. Um diesen zu starten, soll ein Proof-of-Concept erstellt werden. \\
            \\
            Vielen junge Menschen wenden sich ab vom traditionellen Web-Browser und sind stärker in Messenger-Diensten wie WhatsApp, Signal, Telegram und vor Allem auf Social Media Portalen wie Instagram, Facebook, TikTok, SnapChat, etc. zu finden. Um sich auf diese Zielgruppe einzulassen, hat man sich entschieden, einen Bot zu programmieren, der basale Informationen vom Nutzer abfragt und den Bewerber einen Termin vereinbaren lässt. \\
            \\
            Der Instant Messenger Telegram (\url{https://telegram.org/}) ist (neben vielen anderen) besonders unter jungen Menschen, ein beliebtes Kommunikations- und Interaktionsmedium, das Funktionen weit über den einfachen Nachrichtenaustausch hinaus bietet. \\
            \\
            Ziel des Projekts ist es, primär jungen Absolventen die Möglichkeit zu bieten, sich auf eine erste Live Coaching Session vorzubereiten und nicht “völlig blank” in eine bezahlte Beratungsleistung zu investieren, ohne zu wissen, was sie erwartet oder wie man sich vorbereitet.

