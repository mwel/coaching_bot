\chapter{Grundlagen}

Die folgenden Sprachen und Systeme dienen als Grundlage für die im Rahmen dieses Projekts entwickelte Applikation.


\section{Python}

Der größte Teil der Applikation ist in Python 3.8.6 \cite{python3.8.6} geschrieben. Das war zum Stand des Entwicklungsbeginns 2021 die aktuell stabile Python-Version.


\section{Telegram API}

Die Applikation basiert auf der API des Instant Messaging Dienstes Telegram \cite{telegramAPI} sowie deren Extension \cite{telegramAPIext}.
Als Grundgerüst der für den Bot erforderlichen State Machine wurde der ConversationBot von Leandro Toledo et. al. genutzt. \cite{conversationBot} Das Repository enthält eine Vielzahl basaler Bot-Implementierungen, die als Startpunkt für jegliche Bot-Implementierung einen guten Überblick über das Grundgerüst ein die Funktionsweise eines Bot geben.


\section{SQLite}

Zur Speicherung von Nutzerdaten wird eine simple SQLite Datenbank \cite{sqlite} genutzt.


\section{SQLite3 API}

Als Schnittstelle zwischen dem Python basierten Backend und der SQLite Datenbank nutzt die Applikation den sqlite3 Database-Connector \cite{sqlite3API}.


\section{HMTL}

HTML bietet uns die Möglichkeit die grafische Oberfläche in einem Standard Webbrowser auszugeben. \cite{HTML}


\section{PHP}

Die HTML-GUI wird via PHP \cite{php} an die Datenbank gebunden.


\section{Google Calendar API}

Die Applikation bindet die Google Calendar API \cite{googleCalAPI} an, um es dem User zu ermöglichen, einen Termin mit dem Anbieter zu vereinbaren: 


\section{TheCoachingBot}

Schließlich findet sich der gesamte Source-Code inklusive aller Abhängigkeiten in einem öffentlichen GitHub Repository: \cite{repo} 