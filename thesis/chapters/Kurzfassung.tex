\kurzfassung

Anmeldung und Terminvereinbarung für Coaching-Programme sollen automatisiert werden. Gängig wird dies über ein Webformular realisiert. Allerdings erfreut sich diese Herangehensweise keiner großen Beliebtheit und Conversion Rates sind niedrig. Daher wird ein weiterer Kommunikationskanal gesucht, der dabei helfen kann, die Erfolgschancen zu erhöhen. Diese Arbeit behandelt den Versuch, eine Chat-Bot-Technologie zu nutzen, um die Problematik zu lösen. Nach einer detaillierteren Einführung in die Problematik, wird eine Reihe an bereits verfügbaren Systemen analysiert. Diese Analyse gegen Ausschlusskriterien zeigt, dass viele existierende Frameworks für dieses Projekt nicht geeignet sind. Daher wird aus einer Kombination und Erweiterung mehrerer Systeme ein Lösungskonzept für einen geskripteten Konversationsfluss erarbeitet und präsentiert. Das Kapitel Realisierung erläutert technisch abstrakt, wie die verschiedenen Teilsysteme aus dem Konzept angepasst und miteinander verknüpft werden. Im Kapitel Implementierung wird detailliert erklärt, wie genau der Konversationsfluss mit angebundenen Systemen umgesetzt wurde. Relevante Ausschnitte aus dem Programmcode werden in Abschnitten vorgestellt und erklärt. Einige Teile der Applikation sind stärker beleuchtet als Andere. Auf zentrale Funktionen und komplexere Logiken und Systemzusammenhänge wird genauer eingegangen. Da der Bot von interessierten Entwicklern mit wenigen Vorkenntnissen genutzt und weiterentwickelt werden soll, wird versucht, ein Verständnis für diese Zielgruppe herzustellen, das hinreichend ist, um das System betreiben und anpassen zu können. \\
Nach einer Führung durch die laufende Applikation aus Nutzersicht und einigen visuellen Beispielen für die Anwendung und Interaktion wird auf potenzielle Anwendungsszenarien eingangen, die neben dem Hauptanwendungsfall weiteres Potenzial der Applikation erahnen lassen. \\
Die Arbeit schließt mit einer Zusammenfassung und einem Ausblick für weitere, mögliche Entwicklungsstufen und Verbesserungsvorschläge für die Applikation. 

\kurzfassungEN

Registration and the setup of appointments for coaching programs should be automated. This is usually realized via a web forms. However, this approach is rather unsuccessful and conversion rates are low. Therefore, an additional communication channel is needed that can help increase chances of success. This paper deals with the attempt to use a chat bot technology to solve the problem. After a more detailed introduction to the problem, a number of already available systems are analyzed. This analysis against exclusion criteria shows that many frameworks are not suitable for this project. Therefore a solution concept is developed and presented from a combination and extension of several systems. The realization chapter explains on a technically abstract level, how the system parts will be adapted, behave and how they are interconnected. A detailed explanaition of how exactly the system works and relevant excerpts from the program code are presented and explained in the implementation chapter. Selcted snippets are shown - central functions and more complex logic as well as system interrelationships are presented in more detail. Since the bot is intended to be used and further developed by interested developers with little previous knowledge, an attempt is made to create an understanding that is sufficient to enable this group to operate and adapt the system. \\
After a guided tour through the running application from the user's point of view and some visual examples of the application and interaction, potential use cases and scenarios will be discussed, providing an idea of further potential for the application. \\
The paper concludes with a summary and an outlook for further, possible development stages and suggestions for improvements of the application.