\chapter{Einleitung und Problemstellung} \label{Einleitung und Problemstellung}
        
    Viele junge Menschen die nach einem abgeschlossenen Studium in die Arbeitswelt einsteigen, haben keine oder wenig Erfahrung damit, wie sie sich vorbereiten sollen oder welche Schritte erforderlich sind, um einen erfolgreichen Start zu schaffen. Persönliche Beratungsleistungen und speziell Einzel-Coaching können dabei helfen, sich effektiv vorzubereiten und einen erheblichen Wettbewerbsvorteil bieten, sind aber für Berufseinsteiger ohne signifikante finanzielle Mittel meist weder zugänglich noch erschwinglich. Auch Professionals mit hinreichenden Mitteln haben oft Hemmungen und die Einstiegshürde für eine erste Session ist hoch. Diese Hürde soll für beide Zielgruppen herabgesetzt werden. Für eine unkomplizierte und intuitive Ansprache gewinnen digitale, mobile Messenger-Dienste und Social-Media-Kanäle zunehmend an Relevanz. Interaktionen mit jungen Absolventen auf klassischen Websiten stellen sich erfahrungsgemäß vor Allem im persönlichen Networking als wenig erfolgreich heraus.\\
    \\
    Darüber hinaus geht man als Coach mit einem Erstgespräch immer in Vorleistung, weil die erste Kennenlern-Session meist gratis angeboten werden muss, um überhaupt Neukunden zu gewinnen. 
    Aus dem Bedürfnis heraus, mit geringem kontinuierlichen Planungs- und Koordinationsaufwand ein breit gefächertes Klientel im Coaching-Bereich aufzubauen, sollen diverse Kanäle geprüft werden. Ziel ist es, Systeme und Technologien für einen standardisierten Onboarding-Mechanismus und einen entsprechenden Workflow zu finden, um den Prozess bis zum Kennenlernen von manuellem Aufwand soweit als sinnvoll zu abstrahieren und damit verbundene Kosten auf ein Minimum zu reduzieren. Der Standardisierungscharakter ist deshalb sinnvoll, weil eine erste Kontaktaufnahme und Vorbereitung auf eine erste Sitzung erfahrungsgemäß sehr kongruent zueinander oder gar einem Skript folgend verlaufen. \\
    \\
    Coaching ist ein sehr persönliches Geschäftsfeld, in dem ein Web-Formular evtl. nicht den Konversations-Charakter aufweist, den man sich als Nutzer wünscht. Es entsteht eher der Eindruck einer Anmeldung für ein Programm. Ein Coaching ist aber eine Partnerschaft zwischen zwei Individuen. Die Erweiterung der Kommunikationskanäle um einen Chat-Bot wird das Problem nicht abschließend lösen, aber die Hypothese lautet, dass die Hemmschwelle gesenkt werden und höhere Conversions erreicht werden könnten. Die Abwendung vom Web-Browser als Kommunikationsmedium ist keine Neuerung der letzten Jahre und schreitet mit der steten Optimierung von Messenger-Diensten wie WhatsApp, Signal, Telegram und Kommunikationsoptionen via Social-Media-Portalen wie Instagram, Facebook, TikTok, SnapChat, etc. weiter voran. Ein Nebeneinander von Web-Formular und Chat-Bot kann helfen, die Conversion Rate für eine Dienstleistung auf ein neues Niveau zu heben. \cite{conversion} Es soll daher ein Proof of Concept für einen Chat-Bot entwickelt werden.\\ 
    \\ 
    \paragraph{Im Fokus stehen:}
    \begin{enumerate}
        \item Ein Nutzer kann einen Termin mit einem Dienstleister ohne dessen Zutun vereinbaren.
        \item Dem Dienstleister soll die repetitive, manuelle Akquise durch einen geführten, technisch gestützen, Ablauf erleichtert werden. 
        \item Es soll eine Alternative für den Kanal Web-Formular geboten werden. 
        \item Der Kanal soll einen kommunikativen Charakter aufweisen. (wie ein Chat-Bot)
        \item Das System soll durch Anbieter von Coaching-Dienstleistungen mit wenigen Vorkenntnissen leicht adaptierbar sein.
    \end{enumerate}
    
    Ziel des Projekts ist es also, eine erste Version des Coaching Bots zu programmieren, die es ermöglicht, persönliche Angaben zu machen und den Nutzer zu einer Terminvereinbarung hinzuführen. Um den Entwicklungsaufwand so gering wie möglich zu halten, wurde hierzu eine Reihe an Systemen analysiert. Eine detaillierte Aufschlüsselung der Systeme folgt in \ref{Verwandte Arbeiten} Verwandte Arbeiten und \ref{Grundlagen} Grundlagen.




