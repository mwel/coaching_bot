\chapter{Einleitung und Problemstellung}

% Begonnen werden soll mit einer Einleitung zum Thema, also Hintergrund und Ziel erläutert werden.

% Weiterhin wird das vorliegende Problem diskutiert: Was ist zu lösen, warum ist es wichtig, dass man dieses Problem löst und welche Lösungsansätze gibt es bereits. Der Bezug auf vorhandene oder eben bisher fehlende Lösungen begründet auch die Intention und Bedeutung dieser Arbeit. Dies können allgemeine Gesichtspunkte sein: Man liefert einen Beitrag für ein generelles Problem oder man hat eine spezielle Systemumgebung oder ein spezielles Produkt (z.B. in einem Unternehmen), woraus sich dieses noch zu lösende Problem ergibt.

% Im weiteren Verlauf wird die Problemstellung konkret dargestellt: Was ist spezifisch zu lösen? Welche Randbedingungen sind gegeben und was ist die Zielsetzung? Letztere soll das beschreiben, was man mit dieser Arbeit (mindestens) erreichen möchte.


% 1.1 Motivation



% 1.2 Proof of concept für Coaching 

\section{Motivation}
\paragraph{Warum einen Coaching Bot bauen?}
Viele junge Menschen die nach einem abgeschlossenen Studium in die Arbeitswelt einsteigen möchten, haben keine oder wenig Ahnung davon, wie sie sich vorbereiten sollen oder welche Schritte erforderlich sind, um einen erfolgreichen Start zu schaffen. Persönliche Beratungsleistungen und speziell Einzel-Coaching können dabei helfen, sich effektiv vorzubereiten und einen erheblichen Wettbewerbsvorteil bieten, sind aber für Berufseinsteiger ohne signifikante finanzielle Mittel miest weder zugänglich noch erschwinglich.
Das sind in Gründung befindliche Unternehmen -wavehoover consult AG- aus Zürich hat es sich daher zur Aufgabe gemacht, in dieser Hinsicht einen Beitrag zu leisten. Interaktionen mit jungen Absolventen auf klassischen Websiten waren bis dato wenig erfolgversprechend. 
Aus diesem Bedürfnis heraus und aufgrund meiner beruflichen Erkenntnis, dass die erste Kontaktaufnahme und die Vorbereitung auf die erste Sitzung meist sehr ähnlich oder gar einem Skript folgend ablaufen, ist die Idee zu einem standardisierten und automatisierten OnBoarding-Prozess entstanden.
Vielen junge Menschen wenden sich ab vom traditionellen Web-Browser und sind stärker in Messenger-Diensten wie WhatsApp, Signal, Telegram oder vor Allem auf Social Media Portalen wie Instagram, Facebook, TikTok, SnapChat, etc. zu finden. Um sich auf diese Zielgruppe einzulassen, hat man sich entschieden, einen Bot zu programmieren, der Informationen vom User abfragt und
Der Instant Messenger Telegram (https://telegram.org/) ist (neben vielen anderen) besonders unter jungen Menschen, ein beliebtes Kommunikations- und Interaktionsmedium, das Funktionen weit über das einfache Nachrichten-austauschen hinaus abdeckt.
Ziel des Projekts ist es, primär jungen Absolventen die Möglichkeit zu bieten, sich auf eine erste Live Coaching Session vorzubereiten und nicht “völlig blank” in eine bezahlte Beratungsleistung zu investieren, ohne überhaupt bereits zu wissen, was sie erwartet oder wie man sich vorzubereiten hat, um das Meiste für sich selbst heraus zu holen.

