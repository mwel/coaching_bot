\chapter{Beispiel-Kapitel}

In diesem Kapitel wird beschrieben, warum es unterschiedliche Konsistenzmodelle\index{Konsistenzmodelle} gibt. Außerdem werden die Unterschiede zwischen strengen Konsistenzmodellen\index{Linearisierbarkeit} (Linearisierbarkeit, sequentielle Konsistenz)\index{Konsistenz!sequentiell} und schwachen Konsistenzmodellen\index{Konsistenz!schwach} (schwache Konsistenz, Freigabekonsistenz)\index{Freigabekonsistenz} erläutert. Es wird geklärt, was Strenge und Kosten (billig, teuer) in Zusammenhang mit Konsistenzmodellen bedeuten.

\section{Warum existieren unterschiedliche Konsistenzmodelle?}

Laut \cite{Malte:97} sind mit der\index{Replikation} Replikation von Daten immer zwei gegensätzliche Ziele verbunden: die Erhöhung der\index{Verfügbarkeit} Verfügbarkeit und die Sicherung der\index{Konsistenz} Konsistenz der Daten. Die Form der Konsistenzsicherung bestimmt dabei, inwiefern das eine Kriterium erfüllt und das andere dementsprechend nicht erfüllt ist (Trade-off zwischen Verfügbarkeit und der Konsistenz der Daten). Stark konsistente Daten sind stabil, das heißt, falls mehrere Kopien der Daten existieren, dürfen keine Abweichungen auftreten. Die Verfügbarkeit der Daten ist hier jedoch stark eingeschränkt. Je schwächer die Konsistenz wird, desto mehr Abweichungen können zwischen verschiedenen Kopien einer Datei auftreten, wobei die Konsistenz nur an bestimmten Synchronisationspunkten gewährleistet wird. Dafür steigt aber die Verfügbarkeit der Daten, weil sie sich leichter replizieren lassen.

Nach \cite{Mosberger:93} kann die Performanzsteigerung der schwächeren Konsistenzmodelle wegen der Optimierung\index{Optimierung} (Pufferung, Code-Scheduling, Pipelines) 10-40 Prozent betragen. Wenn man bedenkt, dass mit der Nutzung der vorhandenen Synchronisierungsmechanismen schwächere Konsistenzmodelle den Anforderungen der strengen Konsistenz genügen, stellt sich der höhere programmiertechnischer Aufwand bei der Implementierung der schwächeren Konsistenzmodelle als ihr einziges Manko dar.

In \cite{Cheriton:85} ist beschrieben, wie man sich Formen von DSM vorstellen könnte, für die ein beachtliches Maß an\index{Inkonsistenz} Inkonsistenz akzeptabel wäre. Beispielsweise könnte DSM verwendet werden, um die Auslastung von Computern in einem Netzwerk zu speichern, so dass Clients für die Ausführung ihrer Applikationen die am wenigsten ausgelasteten Computer auswählen können. Weil die Informationen dieser Art innerhalb kürzester Zeit ungenau werden können (und durch die Verwendung der veralteten Daten keine großen Nachteile entstehen können), wäre es vergebliche Mühe, sie ständig für alle Computer im System konsistent zu halten \cite{Coulouris:02}. Die meisten Applikationen stellen jedoch strengere Konsistenzanforderungen.

\section{Klassifizierung eines Konsistenzmodells}

Die zentrale Frage, die für die Klassifizierung\index{Konsistenz!streng}\index{Konsistenz!schwach} (streng oder schwach) eines Konsistenzmodells von Bedeutung ist \cite{Coulouris:02}: wenn ein Lesezugriff auf eine Speicherposition erfolgt, welche Werte von Schreibzugriffen auf diese Position sollen dann dem Lesevorgang bereitgestellt werden? Die Antwort für das schwächste Konsistenzmodell lautet: von jedem Schreibvorgang, der vor dem Lesen erfolgt ist, oder in der "`nahen"' Zukunft, innerhalb des definierten Betrachtungsraums, erfolgten wird. Also irgendein Wert, der vor oder nach dem Lesen geschrieben wurde.

Für das strengste Konsistenzmodell, Linearisierbarkeit (atomic consistency), stehen alle geschriebenen Werte allen Prozessoren sofort zur Verfügung: eine Lese-Operation gibt den aktuellsten Wert zurück, der geschrieben wurde, bevor das Lesen stattfand. Diese Definition ist aber in zweierlei Hinsicht problematisch. Erstens treten weder Schreib- noch Lese-Operationen zu genau einem Zeitpunkt auf, deshalb ist die Bedeutung von "`aktuellsten"' nicht immer klar. Zweitens ist es nicht immer möglich, genau festzustellen, ob ein Ereignis vor einem anderen stattgefunden hat, da es Begrenzungen dafür gibt, wie genau Uhren in einem verteilten System synchronisiert werden können.

Nachfolgend werden einige Konsistenzmodelle absteigend nach ihrer Strenge vorgestellt. Zuvor müssen wir allerdings klären, wie die Lese- und Schreibe-Operationen in dieser Ausarbeitung dargestellt werden.

Sei $x$ eine Speicherposition, dann können Instanzen dieser Operationen wie folgt ausgedrückt werden:

\begin{itemize}
	\item $R(x)a$ - eine Lese-Operation\index{Operation!Lesen}, die den Wert $a$ von der Position $x$ liest.
	\item $W(x)b$ - eine Schreib-Operation\index{Operation!Schreiben}, die den Wert $b$ an der Position $x$ speichert.
\end{itemize}

\section{Linearisierbarkeit\index{Linearisierbarkeit} (atomic consistency)}

Die Linearisierbarkeit im Zusammenhang mit DSM kann wie folgt definiert werden:

\begin{itemize}
	\item Die verzahnte Operationsabfolge findet so statt: wenn $R(x)a$ in der Folge vorkommt, dann ist die letzte Schreib-Operation, die vor ihr in der verzahnten Abfolge auftritt, $W(x)a$, oder es tritt keine Schreib-Operation vor ihr auf und $a$ ist der Anfangswert von $x$. Das bedeutet, dass eine Variable nur durch eine Schreib-Operation geändert werden kann.
	\item Die Reihenfolge der Operationen in der Verzahnung ist konsistent zu den \underline{Echtzeiten}\index{Echtzeiten}, zu denen die Operationen bei der tatsächlichen Ausführung aufgetreten sind.
\end{itemize}

Die Bedeutung dieser Definition kann an folgendem Beispiel (Tabelle \ref{tab:1}) nachvollzogen werden. Es sei angenommen, dass alle Werte mit $0$ vorinitialisiert sind.

\begin{table}
	\centering
		\begin{tabular}{l | l l l l}
			\textbf{Prozesse} & \textbf{Zeit} $\rightarrow$ & \\
			\hline
			$P_{1}$ & $W(x)1$ & & $W(y)2$ \\
			$P_{2}$ & & $R(x)1$ & & $R(y)2$ \\
		\end{tabular}
	\caption{Linearisierbarkeit ist erfüllt}
	\label{tab:1}
\end{table}

Hier sind beide Bedingungen erfüllt, da die Lese-Operationen den zuletzt geschriebenen Wert zurückliefern. Interessanter ist es, zu sehen, wann die Linearisierbarkeit verletzt ist.

\begin{table}
	\centering
		\begin{tabular}{l | l l l l}
		\textbf{Prozesse} & \textbf{Zeit} $\rightarrow$ \\
		\hline
		$P_{1}$ & $W(x)1$ & $W(x)2$ \\
		$P_{2}$ & & & \color{red} $R(x)0$ & \color{black} $R(x)2$ \\
		\end{tabular}
	\caption{Linearisierbarkeit ist verletzt, sequentielle Konsistenz ist erfüllt.}
	\label{tab:2}
\end{table}

In diesem Beispiel (Tabelle \ref{tab:2}) ist die Echtzeit-Anforderung verletzt, da der Prozess $P_{2}$ immer noch den alten Wert liest, obwohl er von Prozess $P_{1}$ bereits geändert wurde. Diese Ausführung wäre aber sequentiell konsistent (siehe kommender Abschnitt), da es eine Verzahnung der Operationen gibt, die diese Werte liefern könnte ($R(x)0$, $W(x)1$, $W(x)2$, $R(y)2$). Würde man beide Lese-Operationen des 2. Prozesses vertauschen, wie in der Tabelle \ref{tab:3} dargestellt, so wäre keine sinnvolle Verzahnung mehr möglich.

\begin{table}
	\centering
		\begin{tabular}{l | l l l l}
		\textbf{Prozesse} & \textbf{Zeit} $\rightarrow$ \\
		\hline
		$P_{1}$ & $W(x)1$ & $W(x)2$ \\
		$P_{2}$ & & & \color{red} $R(x)2$ & \color{red} $R(x)0$ \\

		\end{tabular}
	\caption{Linearisierbarkeit und sequentielle Konsistenz sind verletzt.}
	\label{tab:3}
\end{table}

In diesem Beispiel sind beide Bedingungen verletzt. Selbst wenn die Echtzeit, zu der die Operationen stattgefunden haben, ignoriert wird, gibt es keine Verzahnung einzelner Operationen, die der Definition entsprechen würde.
